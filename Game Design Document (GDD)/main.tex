%----------------------- Packages -----------------------

\documentclass{article}
\usepackage[spanish]{babel}
\usepackage{lipsum}
\usepackage{natbib}
\usepackage{graphicx}
\usepackage{analysis_orax}

%----------------------- Document -----------------------

\begin{document}

%----------------------- TitlePage -----------------------

\begin{titlepage}
  
\newlength{\centeroffset}
\setlength{\centeroffset}{-0.5\oddsidemargin}
\addtolength{\centeroffset}{0.5\evensidemargin}
\thispagestyle{empty}
\noindent\hspace*{\centeroffset}

\begin{minipage}{\textwidth}

\centering

\includegraphics[width=0.9\textwidth]{figures/UGR.jpg}\\[2cm]

\textsc{\Large Programación Gráfica de Videojuegos\\[0.5cm]}
\Huge\bfseries Game Design Document\\

\end{minipage}

\vspace{2.5cm}
\noindent\hspace*{\centeroffset}\begin{minipage}{\textwidth}
\centering
\textbf{Autores}\\ [0.25cm] {Bailón Robles, Rafael}\\{Bowen, Kieran Michael}\\{Martín Jáimez, José Adrián}\\{Urrea Echeverry, Juan Carlos}\\{Vendrell Molina, Rafael}\\[3cm]

\includegraphics[width=0.3\textwidth]{figures/ETSIIT.png}\\[0.5cm]

\textsc{Escuela Técnica Superior de Ingenierías Informática y de Telecomunicación}\\
\textsc{---}\\Granada, 10 de noviembre de 2019

\end{minipage}

\end{titlepage}

% %----------------------- TitlePage -----------------------

%----------------------- Indice -----------------------

\pagenumbering{arabic}
\setcounter{page}{2}

\tableofcontents

\clearpage

% %----------------------- Indice -----------------------

%----------------------- Ideas Iniciales -----------------------

\section{1. Ideas Iniciales}

\subsection{1.1. Temática}

\noindent La conquista de un nuevo planeta siempre es dura. Protege tu base, de monstruos gigantes, construyendo torretas. En "The Last Shelter", la humanidad ha llegado a un nuevo planeta habitado por monstruos que no piensan compartir el territorio. Para lidiar con esta situación, y defenderte de las hordas de enemigos, construye tus torres y defiende tus asentamientos. Obten los recursos que necesitas y defiende tu base, culmina la conquista del nuevo mundo.
\\\\
El juego se desarrolla en un espacio continuo y 3D. Las posiciones de las torretas están en espacios discretos 2D, con posiciones fijas en las que se puede colocar, y el camino de las hordas enemigas también será 2D.
\\\\
Hay distintas perspectivas para que el jugador perciba el mundo según la situación:

\begin{itemize}
    \item \textcolor{Orange}{Vista Ortográfica (Proyección Ortogonal)}. Esta vista permite al jugador construir la estrategia de defensa (posicionar las torres). El jugador puede colocar, mejorar, vender y proteger las torres defensivas durante el transcurso del juego.
    \item \textcolor{Orange}{Vista Isométrica}. Esta vista es la otra cámara principal junto con la Vista Ortográfica. Permite realizar las mismas acciones, con una perspectiva diferente.
    \item \textcolor{Orange}{Vista de Torre Defensiva}. Esta vista es sólo para ver lo que la torre defensiva ve. Sirve para observan como llegan los enemigos. La cámara se sitúa sobre las armas de la torre, de forma que cuando se mueva para apuntar, la cámara se mueve también, acompañando el movimiento.
    \item \textcolor{Orange}{Vista de Enemigos}. Esta vista es sólo para ver el terreno desde el punto de vista de los enemigos. Se utiliza para ver como avanzan los enemigos hacia las torres. La cámara se sitúa en primera persona y sigue el movimiento del enemigo seleccionado. 
\end{itemize}

\noindent El jugador es algo exterior al juego, no tiene avatar pero puede crear y manipular distintas torres de varias formas para lograr el objetivo. El jugador controla las torres que defienden la base. Las torres se enfrentan a las hordas de enemigos resultando en la eliminación de los enemigos o en la destrucción de la torre. Otra interacción es el uso de habilidades especiales que afectan al mundo del juego.
\\\\
Fuera de los niveles, el jugador puede conseguir mejoras gastando los puntos de experiencia. También fuera de los niveles, el jugador puede elegir el nivel que quiere jugar.
\\\\
Además, el jugador, puede tomar el control de las armas de la base para utilizarlas como apoyo para las torres defensivas. Es la base la que posee las armas especiales.

\subsection{1.2. Mecánicas del Juego}

\noindent El juego tiene una serie de objetivos. Para completar el objetivo principal de cada nivel el jugador debe sobrevivir a un número determinado de oleadas de enemigos evitando que estos destruyan la base.
\\\\
Para cumplir los objetivos secundarios el jugador debe cumplir ciertos requisitos al terminar el nivel:

\clearpage

\begin{itemize}
    \item Completar el nivel sin recibir daño en la base.
    \item Completar el nivel sin perder más de un número de terminado de torres.
    \item Completar el nivel sin usar una determinada torre.
    \item Completar el nivel sin perder ninguna torre.
    \item Completar el nivel sin usar habilidades especiales.
\end{itemize}

\noindent El principal reto al que se enfrenta el jugador son los enemigos. Debe usar las torres para acabar con ellos y evitar que destruyan la base. También debe lidiar con la posible falta de recursos para construir o mejorar sus torres.

\subsubsection{1.2.1. Acciones}

\noindent \textbf{Acciones del Main Mode:}

\begin{itemize}
    \item \textcolor{Orange}{Jugador}
    \begin{itemize}
        \item Ataque Especial
        \item Construir Torre y Torre de Energía
        \item Controlar Base
        \item Disparar
        \item Localizar Objetivo
        \item Mejorar Torre y Torre de Energía
        \item Poner Cámara en Torre
        \item Poner Escudo en Torre
        \item Reparar Torre
        \item Seleccionar Torre
        \item Vender Torre
    \end{itemize}
    \item \textcolor{Orange}{Enemigos}
    \begin{itemize}
        \item Atacar Torre
        \item Avanzar
        \item Colisión de Enemigos con Torres
        \item Colisión de Enemigos en Distintas Rutas
        \item Colisión de Enemigos en una Misma Ruta
        \item Orientar Enemigo
        \item Poner Cámara en Enemigo
    \end{itemize}
\end{itemize}

\hfill \break \textbf{Acciones del Mapa:}

\begin{itemize}
    \item Acceder al Research Mode
    \item Seleccionar Nivel
    \item Volver al Menú Principal
\end{itemize}

\hfill \break \textbf{Acciones del Research Mode}

\begin{itemize}
    \item Cancelar Mejora
    \item Confirmar Mejora
    \item Elegir Nivel de Mejora
\end{itemize}

\clearpage \textbf{Acciones de Configuración:}

\begin{itemize}
    \item Acelerar
    \item Ralentizar
    \item Reanudar
    \item Reiniciar
    \item Salir
\end{itemize}

\subsubsection{1.2.2. Interacciones}

\begin{itemize}
    \item La derrota de las hordas de enemigos otorgan recursos.
    \item La aparición de un enemigo en el radio de acción de la torre la activa. Las torres tienen enemigos predilectos a los que atacar.
    \item Orientar Arma torre hacia el enemigo seleccionado.
    \item Localizar Objetivo al que disparar por parte de las torres.
    \item Al inicio del nivel se reciben recursos.
    \item Las hordas enemigas se desplazan siguiendo una trayectoria.
    \item Los enemigos atacan a las torres en su trayectoria hasta morir o destruirlas.
    \item La venta de una torre otorga recursos.
    \item Cambio de camara actual a camara en la nuke al dispararla. 
\end{itemize}

\subsubsection{1.2.3. Recursos}

\begin{itemize}
    \item \textcolor{Orange}{Energía}: se incrementa con torres de energía y eliminando enemigos y al comienzo de una oleada además de vender torretas, se gasta en construcción de torretas, mejoras de torretas, uso de habilidades tanto de torreta como propia. No es acumulable para otros niveles.
    \item \textcolor{Orange}{Experiencia}: al cumplir objetivos secundarios y completando niveles, se gasta en mejoras.

\end{itemize}

\subsubsection{1.2.4. Verbos}

\noindent \textbf{Verbos del Jugador:}

\begin{itemize}
    \item Cambio de Cámara
    \item Comenzar Oleada
    \item Construir
    \item Defender
    \item Mejorar
    \item Proteger
    \item Reparar
    \item Seleccionar
    \item Vender
\end{itemize}

\clearpage \textbf{Verbos de las Torres:}

\begin{itemize}
    \item Destruir
    \item Disparar
    \item Matar
    \item Recolectar
\end{itemize}

\hfill \break \textbf{Verbos de los Enemigos:}

\begin{itemize}
    \item Atacar
    \item Avanzar
    \item Destruir
    \item Morir
\end{itemize}

% %----------------------- Ideas Iniciales -----------------------

\cleardoublepage

%----------------------- Entidades y Atributos -----------------------

\section{2. Entidades y Atributos}

\subsection{2.1. Descripción}

\noindent \textbf{Torre}

\hfill \break \noindent Entidad encargada de dañar a los enemigos.

\begin{itemize}
    \item \textcolor{Orange}{MAX\_HEALTH \{INTEGER\}}
    \begin{itemize}
        \item f(Upgrade, Research.Improvements, Type)
        \item Vida Máxima de la entidad Tower.
    \end{itemize}
    \item \textcolor{Orange}{Defensa \{INTEGER\}}
    \begin{itemize}
        \item f(Upgrade, Research.Improvements, Type)
        \item Defensa de la entidad Torre.
    \end{itemize}
    \item \textcolor{Orange}{Health \{0, MAX\_HEALTH\}}
    \begin{itemize}
        \item f(LVL, Research.Improvements , Type)
        \item Vida actual de la entidad Tower.
    \end{itemize}
    \item \textcolor{Orange}{Upgrade \{1, 2, 3\}}
    \begin{itemize}
        \item Actualización de la entidad Tower.
    \end{itemize}
    \item \textcolor{Orange}{Type \{T1, T2, T3\}}
    \begin{itemize}
        \item Tipo de la entidad torre T1 (ligera), T2 (pesada), T3 (zona).
    \end{itemize}
    \item \textcolor{Orange}{Range \{1, MAX\_RANGE\}}
    \begin{itemize}
        \item f(LVL, Research.Improvements, Type)
        \item Distancia a la cual se localiza al objetivo mediante colisión.
    \end{itemize}
    \item \textcolor{Orange}{Area of Effect \{INTERGER\}}
    \begin{itemize}
        \item f(Type)
        \item Área de daño de los disparos.
    \end{itemize}
    \item \textcolor{Orange}{Damage \{1, MAX\_DMG\}}
    \begin{itemize}
        \item f(LVL, Type)
        \item Daño que afecta a la entidad enemigo modificando el valor health del mismo.
    \end{itemize}
    \item \textcolor{Orange}{Rate of fire \{T\_MIN, T\_MAX\}}
    \begin{itemize}
        \item f(LVL, Research.Improvements, Type)
        \item Velocidad con la cual la entidad torre ejecuta la acción de atacar.
    \end{itemize}
    \item \textcolor{Orange}{Appearance \{VISUAL\_MODEL\}}
    \begin{itemize}
        \item f(LVL, Type)
        \item Diseño gráfico de la entidad Tower.
    \end{itemize}
    \hfill \break
    \item \textcolor{Orange}{Sell Price \{INTEGER\}}
    \begin{itemize}
        \item f(LVL, Research.Improvements, Type)
        \item Cantidad de recursos que da la venta el objeto Torre.
    \end{itemize}
    \item \textcolor{Orange}{Buy price \{INTEGER\}}
    \begin{itemize}
        \item f(Type)
        \item Cantidad de recursos que cuesta colocar la torre.
    \end{itemize}
    \item \textcolor{Orange}{Repair price \{INTEGER\}}
    \begin{itemize}
        \item f(Type)
        \item Cantidad de recursos que cuesta reparar la torre.
    \end{itemize}
    \item \textcolor{Orange}{Price \{INTEGER\}}
    \begin{itemize}
        \item f(Type)
        \item Cantidad de recursos que necesita la adquisición del objeto Tower.
    \end{itemize}
    \item \textcolor{Orange}{Camera \{ON, OFF\}}
    \begin{itemize}
        \item Dato que Indica si la entidad Camera está activa.
    \end{itemize}
    \item \textcolor{Orange}{Preferred\_Target \{Bicho.Type\}}
    \begin{itemize}
        \item Dato que indicara el objetivo preferente del objeto Tower.
    \end{itemize}
\end{itemize}

\noindent \textbf{Cámara}

\hfill \break \noindent Entidad encargada de la vista desde distintos objetos del juego, solo uno a la vez.

\begin{itemize}
    \item \textcolor{Orange}{Coordenadas \{X, Y, Z\}}
    \begin{itemize}
        \item Valores que indican la posición del objeto Camera en el mundo del juego.
    \end{itemize}
    \item \textcolor{Orange}{Dirección}
    \begin{itemize}
        \item Valor que indica hacia donde observa el objeto Camera en el mundo del juego.
    \end{itemize}
\end{itemize}

\noindent \textbf{Torre Generadora}

\hfill \break \noindent Entidad encargada de generar recursos periódicamente.

\begin{itemize}
    \item \textcolor{Orange}{MAX\_HEALTH \{INTEGER\}}
    \begin{itemize}
        \item Vida máxima de la entidad Torre Generadora.
    \end{itemize}
    \item \textcolor{Orange}{HEALTH \{0, MAX\_HEALTH\}}
    \begin{itemize}
        \item Vida actual de la entidad Torre Generadora.
    \end{itemize}
    \item \textcolor{Orange}{Sell Price \{MAX\_BENEFITS\}}
    \begin{itemize}
        \item f(LVL, Research.Improvements)
        \item Cantidad de recursos que da la venta del objeto Torre Generadora.
    \end{itemize}
    \clearpage
    \item \textcolor{Orange}{Resources \{INTEGER\}}
    \begin{itemize}
        \item f(Research.Improvements)
        \item Valor que indica la cantidad de recursos que genera el objeto Torre Generadora.
    \end{itemize}
    \item \textcolor{Orange}{Rate\_of\_Resources \{T\_MIN, T\_MAX\}}
    \begin{itemize}
        \item f(LVL, Research.Improvements) 
        \item Velocidad con la cual la entidad Torre Generadora ejecuta la acción de dar recursos. 
    \end{itemize}
    \item \textcolor{Orange}{Appearance \{VISUAL\_MODEL\}}
    \begin{itemize}
        \item Diseño gráfico de la entidad Torre generadora.
    \end{itemize}
\end{itemize}

\noindent \textbf{Base}

\hfill \break \noindent Entidad que el jugador debe proteger para completar el nivel. También puede controlarla para dañar a los enemigos.

\begin{itemize}
    \item \textcolor{Orange}{MAX\_HEALTH \{INTEGER\}}
    \begin{itemize}
        \item Vida máxima de la entidad Base.
    \end{itemize}
    \item \textcolor{Orange}{HEALTH \{0, MAX\_HEALTH\}}
    \begin{itemize}
        \item Vida actual de la entidad Base.
    \end{itemize}
    \item \textcolor{Orange}{Tower\_Play\_Mode \{ON, OFF\}}
    \begin{itemize}
        \item Variable que indica si está activo el modo de control de la base.
    \end{itemize}
    \item \textcolor{Orange}{Range \{1, MAX\_RANGE\}}
    \begin{itemize}
        \item Alcance máximo de la base en modo controlado.
    \end{itemize}
    \item \textcolor{Orange}{DAMAGE \{1, MAX\_DMG\}}
    \begin{itemize}
        \item Daño de la base en modo controlado.
    \end{itemize}
    \item \textcolor{Orange}{Rate\_of\_fire \{T\_MIN, T\_MAX\}}
    \begin{itemize}
        \item Cadencia de fuego de la base en modo controlado.
    \end{itemize}
    \item \textcolor{Orange}{Camera \{ON, OFF\}}
    \begin{itemize}
        \item f(Play\_Mode).
        \item Variable que indica si está activada la cámara de la base.
    \end{itemize}
     \item \textcolor{Orange}{Appearance \{VISUAL\_MODEL\}}
    \begin{itemize}
        \item Diseño gráfico de la entidad Torre generadora.
    \end{itemize}
\end{itemize}

\noindent \textbf{Bicho}

\hfill \break \noindent Entidad de la que el jugador debe defender la base.

\begin{itemize}
    \item \textcolor{Orange}{MAX\_HEALTH \{INTEGER\}}
    \begin{itemize}
        \item Vida máxima de la entidad Bicho.
    \end{itemize}
    \clearpage
    \item \textcolor{Orange}{HEALTH \{0, MAX\_HEALTH\}}
    \begin{itemize}
        \item Vida actual de la entidad Bicho.
    \end{itemize}
    \item \textcolor{Orange}{Resources \{INTEGER\}}
    \begin{itemize}
        \item f(TYPE)
        \item Cantidad de recursos que aporta el bicho al ser eliminado.
    \end{itemize}
    \item \textcolor{Orange}{Damage \{MIN\_DMG, MAX\_DMG\}}
    \begin{itemize}
        \item f(TYPE) 
        \item Daño que inflige el bicho a las unidades aliadas.
    \end{itemize}
    \item \textcolor{Orange}{Speed \{MIN\_SPEED, MAX\_SPEED\}}
    \begin{itemize}
        \item f(TYPE)
        \item Velocidad de desplazamiento del bicho.
    \end{itemize}
    \item \textcolor{Orange}{TYPE \{T1, T2, T3, SB\}}
    \begin{itemize}
        \item Variable que indica el tipo de bicho.
    \end{itemize}
     \item \textcolor{Orange}{Appearance \{VISUAL\_MODEL\}}
    \begin{itemize}
        \item Diseño gráfico de la entidad Bicho.
    \end{itemize}
\end{itemize}

\noindent \textbf{Mejoras}

\hfill \break \noindent Mejoras disponibles para las torres.

\begin{itemize}
    \item \textcolor{Orange}{Effect \{EFFECT\}}
    \begin{itemize}
        \item Efecto que tiene la mejora sobre la torre.
    \end{itemize}
    \item \textcolor{Orange}{Cost \{1, MAX\_COST\}}
    \begin{itemize}
        \item Cantidad de recursos que cuesta la mejora.
    \end{itemize}
     \item \textcolor{Orange}{Appearance \{VISUAL\_MODEL\}}
    \begin{itemize}
        \item Diseño gráfico de la mejora.
    \end{itemize}
\end{itemize}

\noindent \textbf{Mejoras}

\begin{itemize}
        \item \textcolor{Orange}{Init\_resources \{1, MAX\_RESOURCES\}}
    \begin{itemize}
        \item Cantidad de recursos con la que empieza el juego.
    \end{itemize}
    \item \textcolor{Orange}{Pause \{TRUE, FALSE\}}
    \begin{itemize}
        \item Atributo que indica si el tiempo del juego está detenido.
    \end{itemize}
    \item \textcolor{Orange}{Accelerate \{TRUE, FALSE\}}
    \begin{itemize}
        \item Atributo que indica si el tiempo del juego está acelerado.
    \end{itemize}
    \item \textcolor{Orange}{Resources \{0, MAX\_RESOURCES\}}
    \begin{itemize}
        \item Cantidad de recursos que tiene el jugador.
    \end{itemize}
    \clearpage
    \item \textcolor{Orange}{Enemy\_spawn\_point \{List{X, Y}\}}
    \begin{itemize}
        \item Lista de puntos de spawn posibles para los enemigos.
    \end{itemize}
    \item \textcolor{Orange}{Tower\_spawn\_point \{List{X, Y}\}}
    \begin{itemize}
        \item Lista de puntos de spawn posibles para las torres.
    \end{itemize}
     \item \textcolor{Orange}{Collector\_spawn\_point \{List{X, Y}\}}
    \begin{itemize}
        \item Lista de puntos de spawn posibles para las torres recolectoras.
    \end{itemize}
    \item \textcolor{Orange}{Wave\_number \{1, MAX\_WAVES\}}
    \begin{itemize}
        \item Número de oleadas necesarias para completar el nivel.
    \end{itemize}
    \item \textcolor{Orange}{Wave\_started \{TRUE, FALSE\}}
    \begin{itemize}
        \item Indica si hay una oleada activa.
    \end{itemize}
\end{itemize}

\noindent \textbf{Wave}

\hfill \break \noindent Entidad que contiene y gestiona a una oleada particular de bichos.

\begin{itemize}
    \item \textcolor{Orange}{Enemigos \{List(Enemigos, Spawn Enemigo, Tiempo)\}}
    \begin{itemize}
        \item f(Level.Wave\_Number, Level.Enemy\_Spawn\_Point)
        \item Una lista que contiene información sobre cada enemigo que aparecerá
    \end{itemize}
\end{itemize}

\noindent \textbf{Research}

\hfill \break \noindent Entidad encargada de generar recursos periódicamente.

\begin{itemize}
    \item \textcolor{Orange}{EXP \{INTEGER\}}
    \begin{itemize}
        \item f(NIVEL)
        \item La experiencia actual que tiene el jugador
    \end{itemize}
    \item \textcolor{Orange}{Improvements \{List(Tipo, Nivel, EXP)\}}
    \begin{itemize}
        \item Una lista con las mejoras que podría comprar el jugador.
    \end{itemize}
\end{itemize}

\noindent \textbf{Partida}

\hfill \break \noindent Entidad que contiene el progreso que ha hecho el jugador en el juego.

\begin{itemize}
    \item \textcolor{Orange}{Nivel desbloqueado \{1,2,3)\}}
    \begin{itemize}
        \item El nivel hasta el que se ha llegado.
    \end{itemize}
    \item \textcolor{Orange}{Logros \{List(Nivel, Logro)\}}
    \begin{itemize}
        \item Una lista que contiene información sobre los logros que se han obtenido en cada nivel.
    \end{itemize}
    \item \textcolor{Orange}{Nivel superado \{List(dificultad)\}}
    \begin{itemize}
        \item Una lista que contiene información sobre la dificultlad en la que se ha superado cada nivel.
    \end{itemize}
\end{itemize}

\clearpage

\noindent \textbf{HUD}

\hfill \break \noindent Todos los atributos de la IU.

\begin{itemize}
    \item \textcolor{Orange}{Main Mode}
    \begin{itemize}
		\item El HUD durante juego normal.
		\item \textcolor{Orange}{Resources}
		\begin{itemize}
			\item f(Level.resources)
		\end{itemize}
		\item \textcolor{Orange}{EXP}
		\begin{itemize}
			\item f(Research.EXP)
		\end{itemize}
    \end{itemize}
    \item \textcolor{Orange}{Pop-Up Menu Mode}
    \begin{itemize}
        \item Varios menus o iconos que aparecen superimpuestos sobre el juego.
		\item \textcolor{Orange}{Tipo \{Torre, Nivel, Barra de vida\}}
		\item (tipo == Torre)
		\item \textcolor{Orange}{Tower\_Menu}
		\begin{itemize}
			\item Un menu con todas las acciones relativas a una torre disponibles.
			\item (tipo == Barra de vida)
			\item \textcolor{Orange}{HP}
		    \begin{itemize}
			    \item Una barra que muestra la vida actual respecto a la máxima de una torre/enemigo/base.
			    \item f(Torre.HP, Torre.MAX\_HEALTH) || f(Base.HP, Base.MAX\_HEALTH) || f(Enemigo.HP, Enemigo.MAX\_HEALTH)
		    \end{itemize}
		    \item (tipo == Nivel)
		    \item \textcolor{Orange}{Logros}
		    \begin{itemize}
			    \item Los logros que se ha conseguido en cierto nivel.
			    \item f(Partida.Logros)
		    \end{itemize}
		    \item \textcolor{Orange}{Menu\_dificultad}
		    \begin{itemize}
			    \item Permite seleccionar los distintos niveles de dificultad, además de mostrar cuales se han superado.
			    \item f(Partida.Nivel\_superado)
		    \end{itemize}
		\end{itemize}
    \end{itemize}
    \item \textcolor{Orange}{Research Mode}
    \begin{itemize}
		\item El HUD durante la compra de mejoras entre niveles.
		\item \textcolor{Orange}{Mejoras\_menu}
		\begin{itemize}
			\item Las mejoras compradas y disponibles para comprar.
			\item f(Research.Improvements)
		\end{itemize}
    \end{itemize}
	\item \textcolor{Orange}{Map Mode}
    \begin{itemize}
		\item El HUD durante la selección de niveles.
		\item \textcolor{Orange}{Niveles}
		\begin{itemize}
			\item Los niveles seleccionables.
			\item f(Partida.Nivel\_superado)
		\end{itemize}
    \end{itemize}
\end{itemize}

\clearpage

\noindent \textbf{Ataque Especial}

\hfill \break \noindent Entidad encargada de generar recursos periódicamente.

\begin{itemize}
    \item \textcolor{Orange}{TYPE \{Bomb, Nuke\}}
    \begin{itemize}
        \item El tipo de ataque especial.
    \end{itemize}
    \item \textcolor{Orange}{Range \{(0, MAX\_LEVEL)\}}
    \begin{itemize}
        \item f(Research.Improvements ,TYPE)
        \item La distancia a la que se podrá lanzar el ataque especial.
    \end{itemize}
    \item \textcolor{Orange}{Damage \{(0, MAX\_LEVEL)\}}
    \begin{itemize}
        \item f(Research.Improvements ,TYPE)
        \item El daño que hará el ataque especial por defecto.
    \end{itemize}
    \item \textcolor{Orange}{Appearance \{VISUAL\_MODEL\}}
    \begin{itemize}
        \item f(TYPE)
        \item La apariencia visual de la explosión.
    \end{itemize}
    \item \textcolor{Orange}{Area Of Effect \{INTEGER\}}
    \begin{itemize}
        \item f(TYPE)
        \item Área en el que hace daño el ataque.
    \end{itemize}
    \item \textcolor{Orange}{Cooldown \{INTEGER\}}
    \begin{itemize}
        \item f(Research.Improvements ,TYPE)
        \item Cuánto tiempo ha de trascurrir antes de poder usar el ataque especial de nuevo.
    \end{itemize}
\end{itemize}

\clearpage

\subsection{2.2. Estados S-T}

\hfill \break

\begin{figure}[htb]
    \centering
    \includegraphics[width=1\textwidth]{entity-relationship/tower_movimiento.png}
    \centering
    \textcolor{Orange}{\textbf{\caption{Diagrama Entidad-Relacion Torre}}}\label{fig:1}
\end{figure}

\begin{figure}[htb]
    \centering
    \includegraphics[width=1\textwidth]{entity-relationship/bicho_movimiento.png}
    \centering
    \textcolor{Orange}{\textbf{\caption{Diagrama Entidad-Relacion Enemigos}}}\label{fig:2}
\end{figure}

% %----------------------- Entidades y Atributos -----------------------

\cleardoublepage

%--------------------- Acciones, Interacciones y Reglas ---------------------

\section{3. Acciones, Interacciones y Reglas}

\subsection{3.1. Acciones}

\subsubsection{3.1.1. Pop-Up Actions}

\begin{itemize}
    \item \textcolor{Orange}{Vender}
    \begin{itemize}
        \item Limitaciones: vida torre mayor que 0 (no destruida)
        \item Consecuencias: espacio libre, cambio visual (torre desaparece), aumentan recursos
    \end{itemize}
    \begin{figure}[htb]
        \centering
        \includegraphics[width=0.7\textwidth]{action-reaction/sell.png}
        \centering
        \textcolor{Orange}{\textbf{\caption{Diagrama de la Acción Vender}}}\label{fig:3}
    \end{figure}
\end{itemize}

\begin{itemize}
    \item \textcolor{Orange}{Reparar}
    \begin{itemize}
        \item Limitaciones: vida torre mayor que 0 y menor que vida maxima, recursos suficientes
        \item Consecuencias: aumento vida torre, disminución de recursos
    \end{itemize}
    \begin{figure}[htb]
        \centering
        \includegraphics[width=0.7\textwidth]{action-reaction/repair.png}
        \centering
        \textcolor{Orange}{\textbf{\caption{Diagrama de la Acción Reparar}}}\label{fig:4}
    \end{figure}
\end{itemize}

\clearpage

\begin{itemize}
    \item \textcolor{Orange}{Mejorar}
    \begin{itemize}
        \item Limitaciones: tener Torre.Nivel menor que Torre.MAX\_LEVEL, recursos suficientes, mejora desbloqueada
        \item Consecuencias: aumento de Torre.Health, disminución de recursos
    \end{itemize}
    \begin{figure}[htb]
        \centering
        \includegraphics[width=0.7\textwidth]{action-reaction/upgrade.png}
        \centering
        \textcolor{Orange}{\textbf{\caption{Diagrama de la Acción Mejorar}}}\label{fig:5}
    \end{figure}
\end{itemize}

\begin{itemize}
    \item \textcolor{Orange}{Escudo}
    \begin{itemize}
        \item Limitaciones: recursos suficientes, habilidad desbloqueada
        \item Consecuencias: reducción de daño sobre torre, disminución de recursos
    \end{itemize}
    \begin{figure}[htb]
        \centering
        \includegraphics[width=0.7\textwidth]{action-reaction/shield.png}
        \centering
        \textcolor{Orange}{\textbf{\caption{Diagrama de la Acción Escudo}}}\label{fig:6}
    \end{figure}
\end{itemize}

\clearpage

\begin{itemize}
    \item \textcolor{Orange}{Controlar Base}
    \begin{itemize}
        \item Limitaciones: sin limitaciones
        \item Consecuencias: activa el modo de disparo de la base o Tower\_Play\_Mode
    \end{itemize}
    \begin{figure}[htb]
        \centering
        \includegraphics[width=0.7\textwidth]{action-reaction/tower_play_mode.png}
        \centering
        \textcolor{Orange}{\textbf{\caption{Diagrama de la Acción Control Base}}}\label{fig:7}
    \end{figure}
    \begin{figure}[htb]
        \centering
        \includegraphics[width=0.7\textwidth]{action-reaction/shoot_base.png}
        \centering
        \textcolor{Orange}{\textbf{\caption{Diagrama de la Acción Disparar}}}\label{fig:8}
    \end{figure}
\end{itemize}

\clearpage

\subsubsection{3.1.2. HUD Actions}

\begin{itemize}
    \item \textcolor{Orange}{Construir Torre (O Recolector)}
    \begin{itemize}
        \item Limitaciones: recursos suficientes, espacio libre, torre desbloqueada
        \item Consecuencias: cambio visual (nueva torre), espacio ocupado, reducción de recursos
    \end{itemize}
    \begin{figure}[htb]
        \centering
        \includegraphics[width=0.7\textwidth]{action-reaction/click_hud_tower.png}
        \centering
        \textcolor{Orange}{\textbf{\caption{Diagrama de la Acción Click en Icono de Torre}}}\label{fig:9}
    \end{figure}
    \begin{figure}[htb]
        \centering
        \includegraphics[width=0.7\textwidth]{action-reaction/drop_the_tower.png}
        \centering
        \textcolor{Orange}{\textbf{\caption{Diagrama de la Acción Soltar torre}}}\label{fig:10}
    \end{figure}
\end{itemize}

\clearpage

\begin{itemize}
    \item \textcolor{Orange}{Seleccionar Torre}
    \begin{itemize}
        \item Limitaciones: torre seleccionable
        \item Consecuencias: activa menú pop-up
    \end{itemize}
    \begin{figure}[htb]
        \centering
        \includegraphics[width=0.7\textwidth]{action-reaction/click_on_tower.png}
        \centering
        \textcolor{Orange}{\textbf{\caption{Diagrama de la Acción Click en Torre}}}\label{fig:11}
    \end{figure}
\end{itemize}

\begin{itemize}
    \item \textcolor{Orange}{Cámara en Torre o Enemigo}
    \begin{itemize}
        \item Limitaciones: torre o enemigo seleccionable
        \item Consecuencias: cambia la posición de la camara 
    \end{itemize}
    \begin{figure}[htb]
        \centering
        \includegraphics[width=0.7\textwidth]{action-reaction/camera.png}
        \centering
        \textcolor{Orange}{\textbf{\caption{Diagrama de la Acción Camara}}}\label{fig:12}
    \end{figure}
\end{itemize}

\clearpage

\begin{itemize}
    \item \textcolor{Orange}{Ataque Especial}
    \begin{itemize}
        \item Limitaciones: ataque activo (no en tiempo de recarga), ataque desbloqueado
        \item Consecuencias: disminución de la vida de los enemigos, activar tiempo de recarga 
    \end{itemize}
    \begin{figure}[htb]
        \centering
        \includegraphics[width=0.7\textwidth]{action-reaction/bomb.png}
        \centering
        \textcolor{Orange}{\textbf{\caption{Diagrama de la Acción Bomba}}}\label{fig:13}
    \end{figure}
    \begin{figure}[htb]
        \centering
        \includegraphics[width=0.7\textwidth]{action-reaction/nuke.png}
        \centering
        \textcolor{Orange}{\textbf{\caption{Diagrama de la Acción Nuke}}}\label{fig:14}
    \end{figure}
\end{itemize}

\begin{itemize}
    \item \textcolor{Orange}{Pausar el juego}
    \begin{itemize}
        \item Limitaciones:  juego activo
        \item Consecuencias: juego pausado
    \end{itemize}
\end{itemize}

\begin{itemize}
    \item \textcolor{Orange}{Reanudar Juego}
    \begin{itemize}
        \item Limitaciones:  juego pausado
        \item Consecuencias: juego reanudado
    \end{itemize}
\end{itemize}

\begin{itemize}
    \item \textcolor{Orange}{Acelerar Juego}
    \begin{itemize}
        \item Limitaciones: juego en velocidad inferior a la que se active
        \item Consecuencias: juego acelerado
    \end{itemize}
\end{itemize}

\begin{itemize}
    \item \textcolor{Orange}{Ralentizar Juego}
    \begin{itemize}
        \item Limitaciones:  juego en velocidad superior a la que se active
        \item Consecuencias: se ralentiza el juego
    \end{itemize}
\end{itemize}

\subsubsection{3.1.3. Research Actions}

\begin{itemize}
    \item \textcolor{Orange}{Elegir Mejora}
    \begin{itemize}
        \item Limitaciones: tener XP, tener el nivel previo de la mejora si lo hay
        \item Consecuencias: se selecciona la mejora, sin hacerse efectiva 
    \end{itemize}
\end{itemize}

\begin{itemize}
    \item \textcolor{Orange}{Confirmar Mejora}
    \begin{itemize}
        \item Limitaciones: haber seleccionado la mejora, pulsar el botón de aceptación
        \item Consecuencias: se agrega la mejora, se vuelve a la pantalla de mapa
    \end{itemize}
\end{itemize}

\begin{itemize}
    \item \textcolor{Orange}{Cancelar Mejora}
    \begin{itemize}
        \item Limitaciones: pulsar el botón de cancelación
        \item Consecuencias: se vuelve a la pantalla de mapa, no se confirman las mejoras elegidas
    \end{itemize}
\end{itemize}

\subsubsection{3.1.4. Map Actions}

\begin{itemize}
    \item \textcolor{Orange}{Selección de Nivel}
    \begin{itemize}
        \item Limitaciones: nivel desbloqueado
        \item Consecuencias: menú de selección de dificultad
    \end{itemize}
    \item \textcolor{Orange}{Volver al Menú}
    \begin{itemize}
        \item Limitaciones: no tiene
        \item Consecuencias: vuelve al menú principal
    \end{itemize}
    \item \textcolor{Orange}{Acceder al Research}
    \begin{itemize}
        \item Limitaciones: no tiene
        \item Consecuencias: abre el menú research
    \end{itemize}
\end{itemize}

\subsubsection{3.1.5. Difficulty Actions}

\begin{itemize}
    \item \textcolor{Orange}{Seleccionar Dificultad}
    \begin{itemize}
        \item Limitaciones: haber seleccionado un nivel
        \item Consecuencias: accede a la partida en esa dificultad
    \end{itemize}
\end{itemize}

\subsection{3.2. Interacciones}

\begin{itemize}
    \item \textcolor{Orange}{Colisión enemigo-torre}
    \begin{itemize}
        \item Consecuencias:  El enemigo se orienta hacia la torre y comienza a atacar. Mientras ataca, la torre pierde health. Si la torre se queda sin health, muere.
    \end{itemize}
    \item \textcolor{Orange}{Colisión enemigo-base}
    \begin{itemize}
        \item Consecuencias: El enemigo se orienta hacia la base y comienza a atacar. Mientras ataca, la base pierde health. Si la base se queda sin health, el jugador pierde.
    \end{itemize}
    \clearpage
    \item \textcolor{Orange}{Colisión enemigo-enemigo}
    \begin{itemize}
        \item Consecuencias: El enemigo de ID menor tiene prioridad, tanto en la misma vía como en vías distintas.
    \end{itemize}
    \item \textcolor{Orange}{Entrada de un enemigo en la LOS de una torre}
    \begin{itemize}
        \item Consecuencias: La torre, si está disparando ya, evalúa de nuevo a qué enemigo debería estar disparando. Si no, simplemente se pone a disparar al enemigo.
    \end{itemize}
    \item \textcolor{Orange}{Colisión bala/misil/explosión-enemigo}
    \begin{itemize}
        \item Consecuencias: El enemigo pierde health. Si el enemigo se queda sin health, muere y da una cantidad de recursos al jugador.
    \end{itemize}
\end{itemize}

\subsection{3.3. Reglas}

\begin{itemize}
    \item \textcolor{Orange}{Colocar torre}
    \begin{itemize}
        \item Si hay suficientes recursos y no hay ninguna otra torre en el punto seleccionado, coloca una torre.
        \item IF tower.buy\_price < level.resources AND tower\_spawn\_point.is\_empty() AND \break tower\_spawn\_point.isSelected() THEN tower.spawn(tower\_spawn\_point)
    \end{itemize}
    \item \textcolor{Orange}{Reparar las torres}
    \begin{itemize}
        \item Si una torre está dañada y el jugador tiene la suficiente energía, se usa energía para restaurar la salud de la torre.
        \item IF(tower.HEALTH < tower.MAX\_HEALTH) AND (level.resources >= tower.repair\_price) THEN (level.resources -= tower.repair\_price)  AND (tower.HEALTH = tower.MAX\_HEALTH)
    \end{itemize}
    \item \textcolor{Orange}{Mejorar las torres}
    \begin{itemize}
        \item Si la experiencia es suficiente, se puede aumentar el nivel de la torre.
        \item IF (research.XP  >= tower.UPGRADE\_COST) THEN tower.UPGRADE += 1
    \end{itemize}
    \item \textcolor{Orange}{Uso de habilidades}
    \begin{itemize}
        \item Si la habilidad está disponible, el jugador puede utilizar la habilidad.
        \item IF special\_attack.isSelected() AND special\_attack.cooldown == 0 AND \break level.pointClicked(point) THEN special\_attack.launch()
    \end{itemize}
    \item \textcolor{Orange}{Colocar torre generadora}
    \begin{itemize}
        \item Si hay suficientes recursos y no hay ninguna otra torre en el punto seleccionado, coloca una torre.
        \item IF tower.buy\_price < level.resources AND tower\_spawn\_point.is\_empty() THEN \break tower.spawn(tower\_spawn\_point)
    \end{itemize}
    \clearpage
    \item \textcolor{Orange}{Eliminar las torres para recuperar recursos}
    \begin{itemize}
        \item Si la vida de la torre no es 0 y la torre no se está reparando, el jugador podrá eliminar la torre y recuperar recursos.
        \item IF(tower.HEALTH > 0 AND NOT tower.repairing()) THEN (tower.HEALTH = 0) AND \break (level.resources += tower.COST)
    \end{itemize}
    \item \textcolor{Orange}{Saquear enemigo}
    \begin{itemize}
        \item Si un bicho muere y la energía no están al máximo, entonces el jugador gana energía según el tipo de bicho.
        \item IF (bicho.HEALTH == 0) AND (level.resources < MAX\_RESOURCES) THEN \break level.resources += bicho.resources
    \end{itemize}
    \item \textcolor{Orange}{Activar torre}
    \begin{itemize}
        \item Si un bicho entra en el radio de acción de una torre y esta no está atacando o el objetivo es prioritario y el objetivo actual no lo es, entonces esta se activa y ataca al bicho.
        \item IF collision(tower, bicho) THEN IF ((bicho.tipo == tower.preferred\_target AND \break tower.target != tower.preferred\_target) OR tower.target == NULL) THEN \break tower.attack(bicho)
    \end{itemize}
    \item \textcolor{Orange}{Iniciar recursos}
    \begin{itemize}
        \item La partida comienza con una determinada cantidad de energía.
        \item level.resources = level.init\_resources
    \end{itemize}
    \item \textcolor{Orange}{Bicho ataca}
    \begin{itemize}
        \item Si una torre está en el rango de acción de un bicho, este la ataca hasta que la torre es destruida o el bicho muere.
        \item IF collision(bicho, tower) THEN WHILE tower.HEALTH > 0 AND bicho.HEALTH > 0 \break DO attack(tower)

    \end{itemize}
\end{itemize}

% %--------------------- Acciones, Interacciones y Reglas ---------------------

\end{document}

% %----------------------- Document -----------------------